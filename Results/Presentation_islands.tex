% Options for packages loaded elsewhere
\PassOptionsToPackage{unicode}{hyperref}
\PassOptionsToPackage{hyphens}{url}
%
\documentclass[
]{article}
\usepackage{amsmath,amssymb}
\usepackage{lmodern}
\usepackage{ifxetex,ifluatex}
\ifnum 0\ifxetex 1\fi\ifluatex 1\fi=0 % if pdftex
  \usepackage[T1]{fontenc}
  \usepackage[utf8]{inputenc}
  \usepackage{textcomp} % provide euro and other symbols
\else % if luatex or xetex
  \usepackage{unicode-math}
  \defaultfontfeatures{Scale=MatchLowercase}
  \defaultfontfeatures[\rmfamily]{Ligatures=TeX,Scale=1}
\fi
% Use upquote if available, for straight quotes in verbatim environments
\IfFileExists{upquote.sty}{\usepackage{upquote}}{}
\IfFileExists{microtype.sty}{% use microtype if available
  \usepackage[]{microtype}
  \UseMicrotypeSet[protrusion]{basicmath} % disable protrusion for tt fonts
}{}
\makeatletter
\@ifundefined{KOMAClassName}{% if non-KOMA class
  \IfFileExists{parskip.sty}{%
    \usepackage{parskip}
  }{% else
    \setlength{\parindent}{0pt}
    \setlength{\parskip}{6pt plus 2pt minus 1pt}}
}{% if KOMA class
  \KOMAoptions{parskip=half}}
\makeatother
\usepackage{xcolor}
\IfFileExists{xurl.sty}{\usepackage{xurl}}{} % add URL line breaks if available
\IfFileExists{bookmark.sty}{\usepackage{bookmark}}{\usepackage{hyperref}}
\hypersetup{
  pdftitle={Drivers of structure distance decay in a spatial multilayer plant-pollinator network},
  hidelinks,
  pdfcreator={LaTeX via pandoc}}
\urlstyle{same} % disable monospaced font for URLs
\usepackage[margin=1in]{geometry}
\usepackage{longtable,booktabs,array}
\usepackage{calc} % for calculating minipage widths
% Correct order of tables after \paragraph or \subparagraph
\usepackage{etoolbox}
\makeatletter
\patchcmd\longtable{\par}{\if@noskipsec\mbox{}\fi\par}{}{}
\makeatother
% Allow footnotes in longtable head/foot
\IfFileExists{footnotehyper.sty}{\usepackage{footnotehyper}}{\usepackage{footnote}}
\makesavenoteenv{longtable}
\usepackage{graphicx}
\makeatletter
\def\maxwidth{\ifdim\Gin@nat@width>\linewidth\linewidth\else\Gin@nat@width\fi}
\def\maxheight{\ifdim\Gin@nat@height>\textheight\textheight\else\Gin@nat@height\fi}
\makeatother
% Scale images if necessary, so that they will not overflow the page
% margins by default, and it is still possible to overwrite the defaults
% using explicit options in \includegraphics[width, height, ...]{}
\setkeys{Gin}{width=\maxwidth,height=\maxheight,keepaspectratio}
% Set default figure placement to htbp
\makeatletter
\def\fps@figure{htbp}
\makeatother
\setlength{\emergencystretch}{3em} % prevent overfull lines
\providecommand{\tightlist}{%
  \setlength{\itemsep}{0pt}\setlength{\parskip}{0pt}}
\setcounter{secnumdepth}{-\maxdimen} % remove section numbering
\ifluatex
  \usepackage{selnolig}  % disable illegal ligatures
\fi

\title{Drivers of structure distance decay in a spatial multilayer
plant-pollinator network}
\author{}
\date{\vspace{-2.5em}}

\begin{document}
\maketitle

{
\setcounter{tocdepth}{2}
\tableofcontents
}
\hfill\break
\hfill\break
\hfill\break

\hypertarget{distance-decay}{%
\section{Distance decay}\label{distance-decay}}

A well-documented phenomenon in nature is distance decay--- the
decreasing species similarity between two locations as the distance
between them increase. However, studies addressing distance decay in
community structure are rare.\\
~\\
~\\

\hypertarget{goal}{%
\section{Goal}\label{goal}}

We studied how distance affects the modular structure of a multilayer
plant-pollinator network in the Canary Islands. In addition, we
performed null models that explicitly control different components to
disentangle the mechanisms behind distance decay patterns.\\
~\\
~\\

\hypertarget{data}{%
\section{Data}\label{data}}

The study was conducted in The Canary Islands, where six islands and one
location on the mainland were sampled. Distances between locations
ranged from 52 to 450 kilometers. On each location, pollinator-plant
interactions were recorded in two adjacent sites (from 50 to 500 meters
apart) for a total of 14 sites. We aggregated data from any two adjacent
sites on an island or mainland because we were interested in
between-location spatial scale.

\begin{longtable}[]{@{}
  >{\centering\arraybackslash}p{(\columnwidth - 12\tabcolsep) * \real{0.14}}
  >{\centering\arraybackslash}p{(\columnwidth - 12\tabcolsep) * \real{0.07}}
  >{\centering\arraybackslash}p{(\columnwidth - 12\tabcolsep) * \real{0.14}}
  >{\centering\arraybackslash}p{(\columnwidth - 12\tabcolsep) * \real{0.19}}
  >{\centering\arraybackslash}p{(\columnwidth - 12\tabcolsep) * \real{0.12}}
  >{\centering\arraybackslash}p{(\columnwidth - 12\tabcolsep) * \real{0.12}}
  >{\centering\arraybackslash}p{(\columnwidth - 12\tabcolsep) * \real{0.21}}@{}}
\toprule
Location & Scale & Plant richness & Pollinator richness & α - richness &
γ - richness & Number of interactions \\
\midrule
\endhead
Western Sahara & Local & 12 & 60 & 72 & - & 541 \\
Fuerte ventura & Local & 9 & 56 & 65 & - & 598 \\
Gran Canaria & Local & 12 & 50 & 62 & - & 727 \\
Tenerife South & Local & 16 & 68 & 84 & - & 699 \\
Tenerife Teno & Local & 19 & 59 & 78 & - & 1019 \\
Gomera & Local & 15 & 52 & 67 & - & 632 \\
Hierro & Local & 11 & 54 & 65 & - & 511 \\
Canary Islands & Region & 39 & 248 & - & 287 & 4727 \\
\bottomrule
\end{longtable}

\hfill\break
\hfill\break
\hfill\break

\hypertarget{plant-pollinator-multilayer-network}{%
\section{Plant-pollinator multilayer
network}\label{plant-pollinator-multilayer-network}}

Contained four components:

\begin{enumerate}
\def\labelenumi{\arabic{enumi}.}
\item
  Seven layers (six islands and the mainland).
\item
  Two sets of nodes representing pollinator and plant species.
\item
  Intralayer directed weighted links representing pollinator-plant
  interactions within layers.
\item
  Interlayer weighted links connecting any species i to itself between
  two layers. Closer two layers are, the stronger is the interlayer
  link. Ecologically, closer distance increases the likelihood that
  spatial processes such as dispersal occur between two sites\\
  ~\\
  ~\\
\end{enumerate}

\hypertarget{distribution-of-intra-and-interlayer-links}{%
\subsection{Distribution of intra and interlayer
links}\label{distribution-of-intra-and-interlayer-links}}

\hfill\break

\includegraphics[width=1\linewidth,height=0.4\textheight]{Presentation_islands_files/figure-latex/unnamed-chunk-2-1}\\
~\\

\hypertarget{null-models-overview}{%
\section{Null models overview}\label{null-models-overview}}

\hfill\break

\includegraphics[width=800px,height=460px]{null_models}

\hypertarget{island-level}{%
\section{Island level}\label{island-level}}

\hfill\break
\hfill\break

\hypertarget{species-distance-decay}{%
\subsection{Species distance decay}\label{species-distance-decay}}

\hfill\break
We calculated Jaccard similarity of species identity between islands and
tested distance decay using a linear regression model.\\
~\\

\hypertarget{species-distance-decay---empirical-data}{%
\subsubsection{Species distance decay - Empirical
data}\label{species-distance-decay---empirical-data}}

\hfill\break
We observed species distance decay between islands in the empirical data
(\(R^2\) = 0.74, P \textless{} 0.001), which indicates that islands
tended to share less species with increasing distance.\\
~\\

\includegraphics[width=1\linewidth,height=0.4\textheight]{Presentation_islands_files/figure-latex/unnamed-chunk-4-1}

\hfill\break
\hfill\break
\hfill\break

\hypertarget{species-distance-decay--null-model-shuffling-species-between-layers-m_1}{%
\subsubsection{\texorpdfstring{Species distance decay- Null model
shuffling species between layers
(\(M_1\))}{Species distance decay- Null model shuffling species between layers (M\_1)}}\label{species-distance-decay--null-model-shuffling-species-between-layers-m_1}}

\hfill\break
We compared the observed distance decay to that obtained using three
versions of null model where we shuffled species (plants, pollinators
and both) between layers. \(M_1\) changes species labels and interlayer
structure but not intralayer structure.\\
~\\
~\\

Redistributing plant, pollinator and both species among sites did not
break species distance decay (\(R^2_{M_1^P}\) = 0.68, P \textless{}
0.001; \(R^2_{M_1^A}\) = 0.71, P \textless{} 0.001; \(R^2_{M_1^{AP}}\) =
0.35, P = 0.004). The difference was more pronounced when shuffling both
plant and pollinator species, which indicates that both species together
have a stronger effect in distance decay.\\
~\\

\includegraphics[width=1\linewidth,height=0.4\textheight]{Presentation_islands_files/figure-latex/unnamed-chunk-5-1}

\hfill\break
\hfill\break
\hfill\break

\hypertarget{modules-distance-decay}{%
\subsection{Modules distance decay}\label{modules-distance-decay}}

\hfill\break
We calculated distance decay in structure in the same way as for
species, but using module identities. In addition, we used multiple null
models to disentangle the mechanisms behind the pattern found because
differences in structure could emerge due to turnover in species
composition or interaction rewiring.

\hypertarget{modules-distance-decay---empirical-data}{%
\subsubsection{Modules distance decay - Empirical
data}\label{modules-distance-decay---empirical-data}}

\hfill\break
The spatial network was partitioned to 88 modules. Most (85) modules
were found in more than one island, while 3 modules were confined to a
single island. Modules varied in size, ranging from 2 to 44 species,
with an average of 7±1 species per module.\\
~\\

\includegraphics[width=1\linewidth,height=0.4\textheight]{Presentation_islands_files/figure-latex/unnamed-chunk-6-1}

\hfill\break
\hfill\break
We observed modules distance decay in the empirical data (\(R^2\) =
0.67, P \textless{} 0.001), which indicates that islands tended to share
less modules with increasing distance. However, decay in modules was
weaker than for species.\\
~\\

\includegraphics[width=1\linewidth,height=0.4\textheight]{Presentation_islands_files/figure-latex/unnamed-chunk-7-1}

\hfill\break
\hfill\break
\hfill\break

\hypertarget{modules-distance-decay--null-model-shuffling-species-between-layers-m_1}{%
\subsubsection{\texorpdfstring{Modules distance decay- Null model
shuffling species between layers
(\(M_1\))}{Modules distance decay- Null model shuffling species between layers (M\_1)}}\label{modules-distance-decay--null-model-shuffling-species-between-layers-m_1}}

\hfill\break
Similarly to distance decay in species, redistributing plant, pollinator
and both species among islands did not break structure distance decay
(\(R^2_{M_1^P}\) = 0.59, P \textless{} 0.001; \(R^2_{M_1^A}\) = 0.097, P
\textless{} 0.001; \(R^2_{M_1^{AP}}\) = 0.14, P \textless{} 0.001).\\
~\\

\includegraphics[width=1\linewidth,height=0.4\textheight]{Presentation_islands_files/figure-latex/unnamed-chunk-8-1}\\
~\\
~\\
In particular, redistributing pollinators and both plants and
pollinators between islands affected the amount of variation explained
by distance (P \textless{} 0.001); but not the redistribution of plant
species (P = 0.997).\\
~\\

\includegraphics[width=1\linewidth,height=0.4\textheight]{Presentation_islands_files/figure-latex/unnamed-chunk-9-1}

\hfill\break
\hfill\break
\hfill\break

\hypertarget{modules-distance-decay--null-model-shuffling-interactions-within-layers-m_2}{%
\subsubsection{\texorpdfstring{Modules distance decay- Null model
shuffling interactions within layers
(\(M_2\))}{Modules distance decay- Null model shuffling interactions within layers (M\_2)}}\label{modules-distance-decay--null-model-shuffling-interactions-within-layers-m_2}}

\hfill\break
We tested if local structure affects distance decay in structure by
shuffling interactions within layers. \(M_2\) changes the intralayer
structure but conserves the interlayer structure of the network.

Shuffling local structure broke distance decay pattern in structure
(\(R^2_{M_2}\) = 0.12, P = 0.12) and had a significant effect on the
overall structure variation explained by distance (P \textless{}
0.001).\\
~\\

\includegraphics[width=1\linewidth,height=0.4\textheight]{Presentation_islands_files/figure-latex/unnamed-chunk-10-1}\\
~\\

\includegraphics[width=1\linewidth,height=0.4\textheight]{Presentation_islands_files/figure-latex/unnamed-chunk-11-1}

\hfill\break
\hfill\break
\hfill\break

\hypertarget{modules-distance-decay--null-model-shuffling-interactions-between-layers-m_3}{%
\subsubsection{\texorpdfstring{Modules distance decay- Null model
shuffling interactions between layers
(\(M_3\))}{Modules distance decay- Null model shuffling interactions between layers (M\_3)}}\label{modules-distance-decay--null-model-shuffling-interactions-between-layers-m_3}}

\hfill\break
We tested if rewiring interaction across islands affects distance decay
in structure by shuffling interactions of each pair of species between
all the islands in which they co-occur. \(M_3\) shuffles intralayer
links between islands but conserves the interlayer links.

\hfill\break
Shuffling interactions between islands did not break distance decay
pattern in structure (\(R^2_{M_3}\) = 0.33, P = 0.006) but had a
significant effect on the overall structure variation explained by
distance (P \textless{} 0.001).\\
~\\

\includegraphics[width=1\linewidth,height=0.4\textheight]{Presentation_islands_files/figure-latex/unnamed-chunk-13-1}\\
~\\

\includegraphics[width=1\linewidth,height=0.4\textheight]{Presentation_islands_files/figure-latex/unnamed-chunk-14-1}

\hfill\break
\hfill\break
\hfill\break

\hypertarget{modules-distance-decay--beta-diversity-components}{%
\subsubsection{Modules distance decay- Beta diversity
components}\label{modules-distance-decay--beta-diversity-components}}

\hfill\break
Shuffling species between islands deviates more from the empirical value
than shuffling interactions between islands (t-test, \(R^2_{M_3}\)
\textgreater{} \(R^2_{M_2}\), P \textless{} 0.001), which indicates
species turnover is the main driver of distance decay in structure.\\
~\\

\includegraphics[width=1\linewidth,height=0.4\textheight]{Presentation_islands_files/figure-latex/unnamed-chunk-15-1}

\hfill\break
\hfill\break
\hfill\break

\hypertarget{modules-distance-decay--null-model-uniform-value-interlayerlinks-m_4}{%
\subsubsection{\texorpdfstring{Modules distance decay- Null model
uniform value interlayerlinks
(\(M_4\))}{Modules distance decay- Null model uniform value interlayerlinks (M\_4)}}\label{modules-distance-decay--null-model-uniform-value-interlayerlinks-m_4}}

\hfill\break
To test the effect of distance between islands, we fixed the weight of
all interlayer links to a uniform value equal to the median of all the
interlayer weights in the network. \(M_4\) maintains intralayer
structure and the presence of interlayer links.

\hfill\break
Contrary to what we expected, fixing the same distance between islands
produced a similar pattern of distance decay in modules as the empirical
network (\(R^2_{M_4}\) = 0.68, P \textless{} 0.001).\\
~\\

\includegraphics[width=1\linewidth,height=0.4\textheight]{Presentation_islands_files/figure-latex/unnamed-chunk-17-1}\\
~\\
~\\
~\\
~\\
~\\
~\\
~\\
~\\
~\\
~\\
~\\

\end{document}
